% vim: tw=80 sw=2 ts=2 ff=unix spelllang=en spell
\newcommand{\colA}[1]{\begin{columns}[t]\begin{column}{#1}}
\newcommand{\colB}[1]{\end{column}\begin{column}{#1}}
\newcommand{\colEnd}{\end{column}\end{columns}}

\usetheme{CambridgeUS}
\usecolortheme{dolphin}
%\setbeamercovered{transparent}
%\useoutertheme{infolines}

\institute[]{Centrum für Informations- und Sprachverarbeitung (CIS)\\
            Ludwig-Maximilians-Universität München (LMU)\\
            \vspace{1cm}
            \href{https://creativecommons.org/licenses/by-nc-sa/4.0/}%
            {\includegraphics[width=0.2\textwidth]{images/by-nc-sa.eps}}}

% keine Navigationspfeile
\setbeamertemplate{navigation symbols}{} % keine Navigations-Buttons

% Standardschrift verändern für Griechisch
\setsansfont[Mapping=tex-text]{Junicode}
\setmonofont[Mapping=tex-text]{DejaVu Sans Mono}
%\setsansfont[Mapping=tex-text]{Linux Libertine O}
%\setsansfont[Scale=MatchLowercase]{Linux Biolinum O}

% Fußzeile mit Titel und Seitenr.
\definecolor{mygray}{gray}{0.25}
%\setbeamertemplate{footline}[frame number]
\setbeamertemplate{footline}{\color{mygray}\hspace*{2mm}\insertauthor\hfill \insertshorttitle\hfill\insertdate\hspace*{10pt}\insertframenumber\ / \inserttotalframenumber\hspace*{2ex}}

% Schrift für URLs
\definecolor{myblue}{rgb}{0.2 0.0 0.8}

\usepackage{hyperref}
\renewcommand{\UrlFont}{\color{myblue}\footnotesize\sf}
\hypersetup{colorlinks,allcolors=.,urlcolor=blue}

% Bibliography
%\usepackage[backend=biber,style=authoryear,maxcitenames=2,maxbibnames=9]{biblatex}

% Schriftgröße Listings
\RequirePackage{fancyvrb}
\DefineVerbatimEnvironment{Highlighting}{Verbatim}%
  {commandchars=\\\{\},fontsize=\footnotesize}
\DefineVerbatimEnvironment{Verbatim}{Verbatim}%
  {fontsize=\footnotesize}
\newcommand{\pocoto}{\texttt{PoCoTo}}

\title{Datech 2017 -- \pocoto{} Workshop}
\author{Florian~Fink}

\begin{document}

\begin{frame}
	\titlepage
\end{frame}

\section{Introduction to post-correction}
\subsection{Motivation}
\begin{frame}
	In the resent years a lot of historical documents have been
	scanned and OCR'ed.

	\begin{itemize}
		\item The overall quality of the character recognition on historical
			documents is in general good.
		\item The performance of the OCR engines even on historical documents is
			constantly improved.
		\item In some cases the quality can be further improved, by further
			adapting the original images and OCR engines.
		\item But still the quality of the recognition is not good enough for
			deeper scientific studies on the documents.
	\end{itemize}
\end{frame}


\subsection{Text recognition on historical documents}
\begin{frame}
	\centering{\includegraphics[width=\linewidth]{../presentations/images/hist_spellings2.png}}
	Example of the OCR results of a snippet of the \emph{BSB Zedlersches
	Universallexikon: article about salmon.}
\end{frame}

\subsection{Characterise recognition rates}
\begin{frame}
	\begin{table}
		\begin{tabular}{c c c c c}
			Year & Language & ABBYY FR 11.1 & Tesseract 3.03 & OCRopus 0.7 \\\hline
			1544 &   lat.   &    83,14      &     70,32      &      74,59 \\
			1649 &   lat.   &    88,07      &     84,87      &      78,98 \\
			1746 &   dt.    &    97,00      &     91,48      &      95,70 \\
			1779 &   lat.   &    82,13      &     80,77      &      75,46 \\
			1871 &   dt.    &    98,12      &     95,94      &      97,40 \\
		\end{tabular}
	\end{table}

	The results of the text recognition must be manually improved:
	\begin{itemize}
		\item Manual (double) keying of the original sources is expensive.
		\item Interactive postcorrection can be used examine the results of the
			OCR.
		\item Interactive postcorrection can be used to improve the results of the
			OCR.
	\end{itemize}
\end{frame}

\section{\pocoto{} -- The PostCorrectionTool}
\subsection{Overview}
\begin{frame}
	\centering{\includegraphics[width=0.5\linewidth]{../presentations/images/impact.png}}
	\begin{itemize}
		\item \pocoto{} is a tool for the interactive post-correction of OCR'ed
			text:
		\item It was developed as part of the EU founded project IMPACT.
		\item It is open source and hosted on
			\href{https://github.com/cisocrgroup/PoCoTo}{github}.
		\item It contains linguistic and visual aids to support the post-correction.
		\item It contains aids to automatically correct systematic errors in the
			documents.
		\item You find its documentation in the
			\href{https://github.com/cisocrgroup/Resources/blob/master/manuals/pocoto-manual.md}{\pocoto{}
			manual} (included in this workshop's data file).
	\end{itemize}
\end{frame}

\subsection{Features}
\begin{frame}
	\begin{itemize}
		\item \pocoto{} has an automatic update mechanism -- once installed, it is
			automatically kept up to date.
		\item The recognition results are visualized with the images of the original
			documents.
		\item The concordance views enable to examine different errors and error
			pattern over the whole document.
		\item A specialized profiling web-service can be used to get correction
			suggestions for unknown words and frequent error patterns in the document.
		\item Different formats can be read, manually corrected and written back.
	\end{itemize}
\end{frame}

\subsection{Formats}
\begin{frame}
	\pocoto{} supports various formats of different OCR engines:
	\begin{itemize}
		\item hOCR used by Tesseract and Ocropus.
		\item ABBYY-XML used by the ABBYY FineReader.
		\item Ocropus-Directories used by Ocropus.
		\item DocXML used by the language profiler.
	\end{itemize}
\end{frame}

\section{Graphical UI}
\subsection{The main areas}
\begin{frame}
	\pocoto{} is composed by 5 main areas. The size of each area can be
	freely adjusted:
	\begin{enumerate}
		\item The menu area contains various commands for navigation and project
			maintenance.
		\item The main view area shows tokens and offers the main correction
			possibilities.
		\item The complete image area displays the page of the current active
			(selected) token.
		\item The error area lists error frequency lists of common word or
			pattern errors.
		\item The token actions area lets you create concordance views an helps
			you to split and merge tokens.
	\end{enumerate}
\end{frame}

\subsection{The main areas}
\begin{frame}
	\centering{\includegraphics[width=\linewidth]{../presentations/images/pocotomainwindow_marked.png}}
\end{frame}

\subsection{Token-visualization}
\begin{frame}
	\begin{columns}
		\column{.6\textwidth}
		\includegraphics[height=.8\textheight]{../presentations/images/pocoto_ui.png}
		\column{.35\textwidth}
		\begin{itemize}
			\item The token of the text are displayed along with their image details.
			\item The page context shows the active token on the original page.
			\item Error frequencies -- based on the confidence values of the OCR
				engine -- are shown.
		\end{itemize}
	\end{columns}
\end{frame}

\subsection{Concordance-view}
\begin{frame}
	\begin{columns}
		\column{.6\textwidth}
		\includegraphics[height=.8\textheight]{../presentations/images/konkordanz_1.png}
		\column{.35\textwidth}
		\begin{itemize}
			\item Common error patterns in the document can be examined using
				the so-called concordance view.
			\item The concordance view lists similar words and patterns
				encountered in the document.
			\item Consistent error patterns can be easily selected and
				corrected in one step.
		\end{itemize}
	\end{columns}
\end{frame}

\section{Interactive post-correction}
\subsection{Overview}
\begin{frame}
	\begin{itemize}
		\item \pocoto{} supports the correction of tokens.
		\item Split tokens (Splits) can be merged together.
		\item Merged tokens (Merges) can be split.
	\end{itemize}
\end{frame}

\subsection{correcting single tokens}
\begin{frame}
	\begin{columns}
		\column{.6\textwidth}
		\includegraphics<1>[height=.8\textheight]{../presentations/images/correction_1.png}
		\includegraphics<2>[height=.8\textheight]{../presentations/images/correction_2.png}
		\includegraphics<3>[height=.8\textheight]{../presentations/images/correction_3.png}
		\column{.35\textwidth}
		\begin{itemize}
			\item \emph{Suspicious} words are marked in the text.
			\item Words can be marked as correct.
			\item Words can be merged with their right neighbours.
			\item Words can be corrected manually in the window.
		\end{itemize}
	\end{columns}
\end{frame}

\subsection{splits and merges}
\begin{frame}
	\begin{columns}
		\column{.6\textwidth}
		\includegraphics<1>[height=.8\textheight]{../presentations/images/merge_1.png}
		\includegraphics<2>[height=.8\textheight]{../presentations/images/merge_2.png}
		\includegraphics<3>[height=.8\textheight]{../presentations/images/merge_3.png}
		\includegraphics<4>[height=.8\textheight]{../presentations/images/split_1.png}
		\includegraphics<5>[height=.8\textheight]{../presentations/images/split_2.png}
		\includegraphics<6>[height=.8\textheight]{../presentations/images/split_3.png}
		\column{.35\textwidth}
		\begin{itemize}
			\item Merged token can be easily split.
			\item Multiple, split token can be easily merged back together.
		\end{itemize}
	\end{columns}
\end{frame}

\subsection{concordances}
\begin{frame}
	\begin{columns}
		\column{.6\textwidth}
		\includegraphics<1>[height=.8\textheight]{../presentations/images/konkordanz_1.png}
		\includegraphics<2>[height=.8\textheight]{../presentations/images/konkordanz_2.png}
		\includegraphics<3>[height=.8\textheight]{../presentations/images/konkordanz_3.png}
		\includegraphics<4>[height=.8\textheight]{../presentations/images/konkordanz_4.png}
		\includegraphics<5>[height=.8\textheight]{../presentations/images/konkordanz_5.png}
		\column{.35\textwidth}
		\begin{itemize}
			\item Common error patterns in the document can be examined using
				the so-called concordance view.
			\item The concordance view lists similar words and patterns
				encountered in the document.
			\item Consistent error patterns can be easily selected and
				corrected in one step.
		\end{itemize}
	\end{columns}
\end{frame}

\section{\pocoto{} -- Installation}
\subsection{Installation}
\begin{frame}
* You can download the application from
  [this link](http://www.cis.lmu.de/ocrworkshop/pocoto/)
    * After the download has finished you should see a file called
`ocrcorrection.zip` in your download folder.
    * Copy or move this file to a convenient place and extract the
      archive.
    * You will see a folder called `ocrcorrection`.
    * Navigate into the directory `ocrcorrection/bin` and double click
      on the executable file `ocrcorrection` or `ocrcorrection.exe`
    * You can create a link to this executable on your desktop for
      easier access.
    * After you double clicked on the file, \pocoto{} should start.
    * You can create a link to this executable on your desktop for
      easier access.
* The downloading and installation of the tool will be covered in more
depth in the next module
\end{frame}

\subsection{Updating}
\begin{frame}
\end{fame}

\section{\pocoto{} -- Projects and profiles}
\subsection{\pocoto{} project structure}
\begin{frame}
* \pocoto{} handles your input documents as separate projects
* Each project is constructed over a set of different files:
    * The XML output files of your OCR engine.
    * The image input files of your documents -- the same that you
    used for your OCR.
* \pocoto{} expects those files to be organized in a specific way:
    * All the XML files for your project should be in one folder
    * All the image files for your project should be in another
      folder.
    * Each image file should have the same name as its corresponding
      XML
      file^[save for the file's file extension (`.xml`, `.png`, ...)].
* It is more convenient to have the two folders for your XML and image
  files together in one place and use this folder as base path for
  your project.
\end{frame}

	\subsection{\pocoto{}'s file formats}
	\begin{frame}
\pocoto{} understands two different XML file formats, that you can use
to create new projects.
1. The character based Abbyy XML format.
2. The HOCR file format.
\pocoto{} uses the information of the Abbyy XML file format directly
to mark *suspicious* words. It will generate an error frequency list
for you. If you use the HOCR format, \pocoto{} is not able to generate
such an error frequency list for you.
\end{frame}

\subsection{Creating a new project}
\begin{frame}
\centering{\includegraphics<1-2>[height=.4\textheight]{images/newprojectwizardstep1.png}}
\centering{\includegraphics<3>[height=.4\textheight]{images//newprojectwizardstep2.png}}
\centering{\includegraphics<4>[height=.4\textheight]{images/newprojectwizardstep3.png}}
You can create new projects using the project wizard. Click to
`file->create new project` and the first frame of the project wizard
open.
1. Insert a name and a path for your project. Click `next`.
2. Insert a the path of your folder, that contains the XML files and
   select the type of your XML files. Click `next`.
3. Select the path to the folder, that contains your image
files. Click `finish`.
\end{frame}

\subsection{Navigation in the project}
\begin{frame}
\centering{\includegraphics[width=0.8\linewidth]{images/pocototoolbar1.png}}
* After you have created a project, you will see the first page of
  your document opened.
* You can go to other pages, using the buttons in the tool bar.
* You can jump 1, 5 or 10 pages forward or backward at once or go to
  the first or last page of your document.
* You can navigate within a page, using your mouse wheel or the
  scroll bars in the areas.
* You can select or activate single token by simply clicking on them.
* You can increase or decrease the sizes of the different areas using
  your mouse pointer.
\end{frame}

\subsection{Creating a concordance view}
\begin{frame}
  \begin{columns}
    \column{.6\textwidth}
    \includegraphics[width=1.0\textwidth]{images/concordanceview1.png}
    \column{.35\textwidth}
    \begin{enumerate}
    \item You can activate any token and if there exists any similar other
       token you can click to the `show concordance view` button in the
       token action area
    \item You can click on any entry in the two error frequency lists in the
    error area.
    \end{enumerate}
\end{columns}}
\begin{frame}


	\subsection{Profiling a project}
	\begin{frame}
* If you want to profile a document, make sure that you have
  configured a valid profiler web service url (see the
  [profiler manual](https://github.com/cisocrgroup/Resources/blob/master/manuals/profiler-manual.md)
  for more information).
* You can always use the default profiler url of \pocoto{}.
* You can always profile your current project by clicking
  `profiler->order document profiler` in the menu area:
    * If the url is valid and the profiler web service is running, you
      will see a window, which lets you choose which language profile
      to use.
    * Select a language and click to `order document profile`.
    * Do as \pocoto{} says and get your self some coffee.
* After the profiling has stopped, you now will have access to the
  common error pattern tab in the error area and you will get a list
of correction suggestions if you try to correct a token.}
\begin{frame}

	\section{}
	\subsection{}
\begin{frame}
 \Huge Thanks for your attention!
\end{frame}

\end{document}
